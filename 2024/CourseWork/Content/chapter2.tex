% Здесь пишется содержание второй главы
%
\chapter{\MakeUppercase{Машинное обучение на больших данных}}\label{ch:second}

\vspace{\baselineskip}
При необходимости, в главу можно добавить преамбулу с кратким введением в содержание этой главы.
\vspace{\baselineskip}

\section{Задача регресии}
\subsection{Постановка задачи регрессии}\vspace{\baselineskip}

Здесь нужно сформулировать поставленную задачу регрессии, которая будет решаться дальше.

\vspace{\baselineskip}\subsection{Решение задачи регрессии}\vspace{\baselineskip}

Здесь подробно описывается решение задачи регрессии.

\vspace{\baselineskip}\subsection{Анализ полученных результатов}\vspace{\baselineskip}

После решения задачи и получения результатов их необходимо проинтерпретировать.

\vspace{\baselineskip}\section{Задача бинарной классификации}
\subsection{Постановка задачи бинарной классификации}\vspace{\baselineskip}

Аналогично задаче регрессии.

\vspace{\baselineskip}\subsection{Решение задачи бинарной классификации}\vspace{\baselineskip}

Аналогично задаче регрессии.

\vspace{\baselineskip}\subsection{Анализ полученных результатов}\vspace{\baselineskip}

Аналогично задаче регрессии.

\vspace{\baselineskip}\section{Выводы}\vspace{\baselineskip}

Сформулированные выводы по главе.
