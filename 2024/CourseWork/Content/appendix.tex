% Здесь пишется содержание приложений
%
\chapter{Пример листинга программного кода}\label{app:A}\vspace{\baselineskip}

Здесь можно привести полный листинг кода программы или модуля.

\begin{code}
import matplotlib.pyplot as plt
import numpy as np

# Данные для графика
x = np.linspace(0, 10, 100)
y = np.sin(x)

# Создание графика
plt.figure(figsize=(10, 6))

plt.plot(x, y, label='sin(x)', color='blue', linewidth=2)

# Настройка графика
plt.title('График функции sin(x)', fontsize=16)
plt.xlabel('x', fontsize=14)
plt.ylabel('sin(x)', fontsize=14)
plt.legend()
plt.grid(True)

# Вывод графика
plt.show()    
\end{code}

\chapter{Пример второго приложения}\label{app:B}\vspace{\baselineskip}

При необходимости, приложений может быть несколько.
