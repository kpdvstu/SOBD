% Содержание введения
%
Во введении сначала дается краткая характеристика области, в которой выполнена работа (1 -- 3 предложения). Затем обосновывается актуальность работы.

Далее идут фразы, которые лучше повторить дословно:

В связи с этим целью данной работы являлось ... (цель должна быть одна).

Для достижения поставленной цели решались следующие задачи:
\begin{enumerate}
    \item первая задача;
    \item вторая задача;
    \item третья задача;
    \item ...
\end{enumerate}

В конце введения следует добавить описание структуры курсовой работы. Например:

В первом разделе рассмотрена более подробно постановка задачи и проведен обзор литературы по ... Во втором разделе ... В третьем разделе ... В заключении работы сформулированы общие выводы ...